\documentclass[12pt]{article}
\usepackage[swedish]{babel}
\usepackage[applemac]{inputenc}

\usepackage{full page} %makes the allover margin spaces smaller


% F�r att f� grafik att fungera.
\usepackage{epsf}
\usepackage{psfrag}
\usepackage{epstopdf}
\usepackage{subfigure}
\usepackage{graphicx}          
%%%%


\title{\center Quadracopter project}			% used by \maketitle

\author{ Mats Malmberg, 860802-0338, matsmalmberg86@gmail.com}

\begin{document}
\maketitle			% automatic title!

\section{Project plan}

\subsection{Milestones}
This is a list of the milestones for the project. Most of them can and will be executed in parallell.
\begin{itemize}
\item[MS1] Technical specification on hardware (based on soft constraints), along with suppliers
\item[MS2] Assembled hardware, ready to install operative system
\item[MS3] Create an embedded linux kernel, that complies with the hardware
\item[MS4] Implemented test software that can access and control all necessary hardware in every intended way
\item[MS5] Implement control software to hover in place
\end{itemize}

\section{Technical documentation}
loose specifications:\\
\begin{itemize}
\item being able to fly horizontaly through door openings
\item being able to manouver with an external load of 2 kg
\item fast wireless communication
\item possibility to extend unit with a robotic arm
\item possibility to extend unit with autonomous navigation and manouverability
\item awesome looks
\item speaker
\item possibility to extend unit with computer vision
\item operating on embedded linux
\item usb port
\item reasonable air time
\item easy to exchange battery pack
\end{itemize}

\section{Documentation}
loggins: \\
root: root\\
developer: developer\\
\subsection{packages post installed}
\begin{itemize}
\item full system update "pacman -Syu"
\item sudo "pacman -S sudo"
\item gcc "pacman -S gcc"
\item unzip "pacman -S unzip"
\item downloaded pigpio library. http://abyz.co.uk/rpi/pigpio/download.html
\item make "pacman -S make"
\end{itemize}





\end{document}
